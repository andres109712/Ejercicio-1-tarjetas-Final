\documentclass{article}
\usepackage[utf8]{inputenc}
\usepackage[spanish]{babel}
\usepackage{listings}
\usepackage{graphicx}
\graphicspath{ {images/} }
\usepackage{cite}

\begin{document}

\begin{titlepage}
    \begin{center}
        \vspace*{1cm}
            
        \Huge
        \textbf{Ejercicio tarjetas}
            
        \vspace{0.5cm}
        \LARGE
        Primera actividad informatica II
            
        \vspace{1.5cm}
            
        \textbf{Andres Camilo Agudelo cadavid }
            
        \vfill
            
        \vspace{0.8cm}
            
        \Large
        Despartamento de Ingeniería Electrónica y Telecomunicaciones\\
        Universidad de Antioquia\\
        Medellín\\
        Marzo de 2021
            
    \end{center}
\end{titlepage}

\tableofcontents
\newpage
\section{Sección introductoria}\label{intro}
esta actividad esta guiada a mostrar como podemos plasmar el proceso de pensamiento logico y paso a paso de una actividad similar a la de programar, independiente de el lenguaje.

\section{Sección de contenido} \label{contenido}
en este ejercicio, se procedio a darle el paso a paso el cual va a continuacion:
-Tener a la mano y en una superficie plana dos tarjetas (pueden ser células, tarjetas de crédito o semejantes) y una hoja de papel 
-La superficie debe estar completamente limpia y sin objetos tanto por debajo como por encima de la hoja
- Solo puedes utilizar una sola mano, sea tanto la izquierda o la derecha. Usa la que más prefieras
- Vas a llevar las dos tarjetas encima de la hoja 
- Agarras la tarjeta que este encima y vas a poner el dedo meñique por debajo de ella
- Ahora, agarras la segunda, de tal forma que tu dedo meñique separe la una de la otra
- pones las tarjetas en una orientación vertical, y con ayuda de tu dedo índice apoyaras las tarjetas en la hoja
-Ahora con tu meñique, separaras los extremos de las tarjetas formando así con ellas un triángulo 
- Sueltalas cuando sientas que las tarjetas se puedan quedar solas en esa posición
dado esto en el video se puede observar el como las personas las cuales hicieron el reto, usaron variantes, obviaron pasos o los realizaron correctamente pero de una manera mas agil o con mayor efectividad, lo que nos da a ver que cada persona tiene una forma distinta de ver un problema, seguir un paso a paso o llegar a una solucion independiente o no.

.











En la Figura (\ref{fig:cpplogo}), se presenta el logo de C++ contenido en la carpeta images.

\begin{figure}[h]
\includegraphics[width=4cm]{cpplogo.png}
\centering
\caption{Logo de C++}
\label{fig:cpplogo}
\end{figure}



\bibliographystyle{IEEEtran}


\end{document}
